%%%%%%%%%%%%%%%%%%%%%%%%%%%%%%%%%%%%%%%%%%%%%%%%%%%%%
% Gugler Labs, GNU/Linux, Trabajo Practico Final
% Segundo cuatrimestre 2017
%
% octubre 2017, rev 1
%%%%%%%%%%%%%%%%%%%%%%%%%%%%%%%%%%%%%%%%%%%%%%%%%%%%%

\documentclass[10pt,a4paper]{article}
\usepackage[utf8]{inputenc}
\usepackage[spanish]{babel}

\usepackage{graphicx}
\usepackage{hyperref}
\usepackage{booktabs}
\usepackage{placeins}

\usepackage{fancyhdr}
\setlength{\headheight}{15pt}
\pagestyle{fancy}
\lhead{}
%\chead{\includegraphics[scale=0.5]{./banner.jpg}}
\rhead{}
\cfoot{P\'agina \thepage}

\author{Leandro Torres}
\title{\Huge{Trabajo Pr\'actico Final}\\
	\vspace{5cm}
	\huge{systemd a trav\'es de un caso de estudio}}
%\date{}
\begin{document}

\maketitle
\pagebreak
\tableofcontents

\section{Introducci\'on}

Un cliente particular necesita llevar un control de una remodelaci\'on que se lleva a cabo en las instalaciones de la empresa en la que trabaja. Necesita tomar una fotograf\'ia cada hora durante dos semanas. Como no va a estar en el pa\'is durante 45 d\'ias necesita verlas desde cualquier parte del mundo. S\'olo tenemos acceso a un toma de 220V y un puerto ethernet con una conexi\'on de 2MB sim\'etrico. Las pol\'iticas de la empresa no permiten la apertura de puertos especiales, de modo que no podr\'iamos acceder remotamente a trav\'es de un servicio SSH; pero podr\'iamos tener acceso f\'isico en horarios de oficina. Por \'ultimo cabe destacar que los servicios de internet y electricidad suelen sufrir cortes espor\'adicos.\\

Se ofrece al cliente colocar un m\'odulo \emph{Raspberry Pi 2 B+} (en adelante \emph{rpi2}) con una c\'amara anexada. Este m\'odulo se alimenta desde un cargador de bater\'ia para celular (a modo de UPS) conectado a la red de 220V. La conexi\'on a internet se realiza mediante el conexionado al puerto ethernet.\\

Cada 60 minutos la rpi2 toma una fotograf\'ia de la obra y la guarda en un directorio alojado en el ra\'iz de un pendrive conectado a la rpi2 para tal fin (la capacidad supera en dos veces los requerimientos). Cada 6 horas la rpi2 sincroniza este directorio con uno en la cuenta \emph{gmail} abierta para tal fin (se utiliza el servicio \emph{Drive} de \emph{GMail}).

\section{Preparaci\'on}

En primer lugar debemos preparar el \emph{hardware}, esto es:
\begin{itemize}
    \item rpi2
    \item Raspbian
    \item alimentaci\'on
    \item c\'amara
\end{itemize}

\subsection{rpi2}

No es el prop\'osito de este trabajo ahondar sobre las caracter\'isticas de una rpi2, bastar\'a una breve descripci\'on de las capacidades que nos interesa.\\

Una rpi2 es una computadora de bajo costo del tama\~no de una tarjeta de cr\'edito. Existen diferentes modelos, para este proyecto necesitamos el que tiene un puerto ethernet y al menos un puerto USB. El sistema operativo corre desde una tarjeta microSD\footnote{Con una capacidad de $4 GB$ estamos cubiertos.}.\\

A continuaci\'on una breve enumeraci\'on de las caracter\'isticas t\'ecnicas:
\begin{description}
    \item[CPU] $900MHZ$ quad-core ARM Cortex-A7
    \item[RAM] $1GB$
    \item[USB] $4$ puertos
    \item[ethernet] $1$ puerto de $100MB$
    \item[Video] puerto HDMI
    \item[Audio] jack stereo de $3.5mm$
    \item[interface] una para la c\'amara y otra para un display, adem\'as de un pinout GPIO.
\end{description}

\subsection{Raspbian}

Si decimos que una rpi2 es una \emph{computadora} podemos suponer que necesita un sistema operativo\footnote{Y hacemos bien en sospechar que se trata de una distribuci\'on GNU/Linux.}. En la p\'agina oficial de Raspberry Pi se encuentran disponibles varias alternativas, la que nos interesa es \emph{Raspbian}.\\

Raspbian es el sistema operativo oficial de la Fundaci\'on Raspberry Pi. Es un sistema operativo basado en \emph{Debian} y compilado para correr en una rpi2\footnote{El SO es tan completo y al mismo tiempo tan liviano que existe una versi\'on para PC, \emph{Raspbian Pi Desktop}.}.\\

Al momento de escribir este informe la \'ultima versi\'on disponible es de finales de junio de 2018, la versi\'on del kernel es 4.14. La \emph{versi\'on} de Raspian es \emph{Stretch}\footnote{En opini\'on del que escribe, el planeta Raspberry pertenece al universo Debian.}. Si bien la opci\'on popular es Raspbian con entorno gr\'afico vamos a optar por la versi\'on \emph{Lite}\footnote{No es otra cosa que el sistema base, lo que en la jerga se conoce como \emph{un debian pelado}.}.\\

La \emph{instalaci\'on} se trata de \emph{copiar} el sistema Raspbian en la tarjeta microSD. Dicho as\'i no s\'olo es una sobresimplificaci\'on del procedimiento, sino tambi\'en de un error de conceptos; pero si lo pensamos bien \emph{todo} en un sistema GNU/Linux es un \emph{archivo}; y a diferencia de una PC, donde encontramos distintos dispositivos con diferentes \emph{firmwares}\footnote{En el universo Windows se los conoce como \emph{drives}.} una rpi2 es id\'entica a otra rpi2 que podemos conseguir en un local de Bangladesh, por lo que s\'olo se necesita compilar el sistema operativo una vez, crear la imagen y hacerla accesible para cualquiera que necesite clonarla.\\

\section{Servicios}

Los servicios ser\'an expuestos s\'olo en su objetivo, no describiremos el c\'odigo por muy simple que sea.
\begin{itemize}
    \item tomar foto
    \item sincronizar carpeta con fotos
    \item log
\end{itemize}

\section{Timers}

Los servicios son disparados por \emph{systemd}.

\section{Servicios al inicio}

Hay servicios que deben ejecutarse al inicio, esto es, cada vez que la rpi2 bootee.

\section{Conclusi\'on}

La conclusi\'on es que systemd es m\'as complicado pero no nos queda otra.

\end{document}
