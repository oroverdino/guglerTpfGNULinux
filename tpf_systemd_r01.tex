%%%%%%%%%%%%%%%%%%%%%%%%%%%%%%%%%%%%%%%%%%%%%%%%%%%%%
% Gugler Labs, GNU/Linux, Trabajo Practico Final
% Segundo cuatrimestre 2017
%
% octubre 2017, rev 1
%%%%%%%%%%%%%%%%%%%%%%%%%%%%%%%%%%%%%%%%%%%%%%%%%%%%%

\documentclass[10pt,a4paper]{article}
\usepackage[utf8]{inputenc}
\usepackage[spanish]{babel}
\usepackage{siunitx}

\usepackage{hyperref}
\usepackage{booktabs}
\usepackage{placeins}
\usepackage[table,xcdraw]{xcolor}

\usepackage{framed,color}
\definecolor{shadecolor}{rgb}{1,1,0.9}
\usepackage{mdframed}[backgroundcolor=yellow]

\usepackage{listings}
\lstset{
    breaklines=true,
    basicstyle=\footnotesize,
    frame=leftline,
    texcl=true,
    basicstyle=\ttfamily,
    escapechar=!
}
\usepackage{graphicx}
\graphicspath{ {./pictos/} }

\usepackage{fancyhdr}
\setlength{\headheight}{15pt}
\pagestyle{fancy}
\lhead{}
%\chead{\includegraphics[scale=0.5]{./banner.jpg}}
\rhead{}
\cfoot{P\'agina \thepage}

\author{Leandro Torres}
\title{\Huge{Trabajo Pr\'actico Final}\\
	\vspace{5cm}
	\huge{systemd a trav\'es de un caso de estudio}}
%\date{}
\begin{document}

\maketitle
\pagebreak
\tableofcontents

\section{Introducci\'on}

Un cliente particular necesita llevar un control de una remodelaci\'on que se lleva a cabo en las instalaciones de la empresa en la que trabaja. Necesita tomar una fotograf\'ia cada hora durante dos semanas. Como no va a estar en el pa\'is durante 45 d\'ias necesita verlas desde cualquier parte del mundo. S\'olo tenemos acceso a un toma de 220V y un puerto ethernet con una conexi\'on de 2MB sim\'etrico. Las pol\'iticas de la empresa no permiten la apertura de puertos especiales, de modo que no podr\'iamos acceder remotamente a trav\'es de un servicio SSH; pero podr\'iamos tener acceso f\'isico en horarios de oficina. Por \'ultimo cabe destacar que los servicios de internet y electricidad suelen sufrir cortes espor\'adicos.\\

Se ofrece al cliente colocar un m\'odulo \emph{Raspberry Pi 2 B+} (en adelante \emph{rpi2}) con una c\'amara anexada. Este m\'odulo se alimenta desde un cargador de bater\'ia para celular (a modo de UPS) conectado a la red de \SI{220}{V}. La conexi\'on a internet se realiza mediante el conexionado al puerto ethernet.\\

Cada \SI{60}{min} la rpi2 toma una fotograf\'ia de la obra y la guarda en un directorio alojado en el ra\'iz de un pendrive conectado a la rpi2 para tal fin. Cada \SI{6}{h} la rpi2 sincroniza este directorio con uno en la cuenta \emph{gmail} abierta para tal fin (se utiliza el servicio \emph{Drive} de \emph{GMail}).

\section{Preparaci\'on}

En primer lugar debemos preparar el \emph{hardware}, esto es:
\begin{itemize}
    \item rpi2
    \item Raspbian
    \item alimentaci\'on
    \item c\'amara
    \item memoria USB
\end{itemize}

\subsection{rpi2}

No es el prop\'osito de este trabajo ahondar sobre las caracter\'isticas de una rpi2, bastar\'a una breve descripci\'on de las capacidades que nos interesa.\\

Una rpi2 es una computadora de bajo costo del tama\~no de una tarjeta de cr\'edito. Existen diferentes modelos, para este proyecto necesitamos el que tiene un puerto ethernet y al menos un puerto USB. El sistema operativo corre desde una tarjeta microSD\footnote{Con una capacidad de \SI{4}{GB} estamos cubiertos.}.\\

A continuaci\'on una breve enumeraci\'on de las caracter\'isticas t\'ecnicas:
\begin{description}
    \item[CPU] \SI{900}{MHz} quad-core ARM Cortex-A7
    \item[RAM] \SI{1}{GB}
    \item[USB] $4$ puertos
    \item[ethernet] $1$ puerto de \SI{100}{MB}
    \item[Video] puerto HDMI
    \item[Audio] jack stereo de \SI{3,5}{mm}
    \item[interface] una para la c\'amara y otra para un display, adem\'as de un pinout GPIO.
\end{description}

\subsection{Raspbian}

Si decimos que una rpi2 es una \emph{computadora} podemos suponer que necesita un sistema operativo\footnote{Y hacemos bien en sospechar que se trata de una distribuci\'on GNU/Linux.}. En la p\'agina oficial de Raspberry Pi se encuentran disponibles varias alternativas, la que nos interesa es \emph{Raspbian}.\\

Raspbian es el sistema operativo oficial de la Fundaci\'on Raspberry Pi. Es un sistema operativo basado en \emph{Debian} y compilado para correr en una rpi2\footnote{El SO es tan completo y al mismo tiempo tan liviano que existe una versi\'on para PC, \emph{Raspbian Pi Desktop}.}.\\

Al momento de escribir este informe la \'ultima versi\'on disponible es de finales de junio de 2018, la versi\'on del kernel es 4.14. La \emph{versi\'on} de Raspian es \emph{Stretch}\footnote{En opini\'on del que escribe, el planeta Raspberry pertenece al universo Debian.}. Si bien la opci\'on popular es Raspbian con entorno gr\'afico vamos a optar por la versi\'on \emph{Lite}\footnote{No es otra cosa que el sistema base, lo que en la jerga se conoce como \emph{un debian pelado}.}.\\

La \emph{instalaci\'on} se trata de \emph{copiar} el sistema Raspbian en la tarjeta microSD. As\'i expresado es una sobresimplificaci\'on del procedimiento y un error de conceptos; pero bien pensado \emph{todo} en un sistema GNU/Linux es un \emph{archivo}; y a diferencia de una PC, donde encontramos distintos dispositivos con diferentes \emph{firmwares}\footnote{En el universo Windows se los conoce como \emph{drives}.} una rpi2 es id\'entica a otra rpi2 que podemos conseguir en un local de Bangladesh, por lo que s\'olo se necesita compilar el sistema operativo una vez, crear la imagen y hacerla accesible para cualquiera que necesite clonarla.\\

En el siguiente c\'odigo vemos c\'omo listamos los dispositivos conectados buscando la microSD, desmontamos y clonamos el sistema operativo\footnote{Algunos comandos necesitan derechos de \emph{root}.}. La imagen que provee el sitio ofical est\'a comprimida en un archivo \emph{zip} que se descarg\'o en el directorio \emph{/tmp} y se la descomprimi\'o ah\'i\footnote{Bajo \emph{systemd} el directorio /tmp es montado autom\'aticamente con un sistema de archivo \emph{tmpfs}, esto es, en la memoria RAM del sistema.}.\\

\begin{scriptsize}
\begin{lstlisting}
!\#! blkid -o list
!\#! umount /media/leandro/0403-0201
!\#! 7z x /tmp/2018-06-27-raspbian-stretch-lite.zip
!\#! dd bs=4M if=/tmp/2018-06-27-raspbian-stretch-lite.img of=/dev/mmcblk0 status=progress
\end{lstlisting}
\end{scriptsize}

\subsection{Energ\'ia}

Este proyecto necesita una fuente de alimentaci\'on que resuelva el problema de los cortes espor\'adicos de energ\'ia. Podr\'ia emplearse una UPS\footnote{\emph{Uninterruptible Power Supply.}} pero con un cargador de celular que disponga de al menos una salida de energ\'ia \emph{mientras se est\'a cargando} resuelve el problema.\\

No ahondaremos aqu\'i sobre los detalles del \emph{power bank}, el \'unico dato importante es la potencia que debe poder entregar en el momento de m\'aximo consumo, que es cuando la rpi2 bootea y cuando graba una fotograf\'ia en el USB (Cuadro \ref{tab:energia}).

\begin{table}[h!]
    \begin{center}
        \begin{tabular}{llL}
        \cline{1-2}
        \multicolumn{1}{|l|}{\cellcolor[HTML]{EFEFEF}Situaci\'on}   & \multicolumn{1}{r|}{\cellcolor[HTML]{EFEFEF}Corriente}    & \\
        \cline{1-2}
        \multicolumn{1}{|l|}{boot + USB}                            & \multicolumn{1}{r|}{900-1400 mAh}                         & \\
        \cline{1-2}
        \multicolumn{1}{|l|}{ocioso (\emph{idle}) + USB}            & \multicolumn{1}{r|}{960 mAh}                              & \\
        \cline{1-2}
        \multicolumn{1}{|l|}{CPU con una carga al 400\% + USB}      & \multicolumn{1}{r|}{1250 mAh}                             & \\
        \cline{1-2}
        \multicolumn{1}{|l|}{idle + c\'amara + USB}                 & \multicolumn{1}{r|}{1200 mAh}                             &
        \end{tabular}
    \end{center}
    \caption{Consumo aproximado}
    \label{tab:energia}
\end{table}

\subsection{C\'amara}

Los creadores de rpi2 dispusieron de una interfase exclusiva para una c\'amara modular, que se conecta mediante un cable plano dise\~nado para romperse con facilidad (Figura \ref{fig:ribbon}). Se compra, se conecta y funciona luego de habilitar el puerto. Consume alrededor de $250 mAh$ desde el momento en que es habilitada (para el caso, desde el booteo); \'este es un factor a tener en cuenta porque suele creerse que s\'olo consume energ\'ia cuando captura una imagen o graba un video, lo cierto es que hay un leve aumento de consumo pero es el proceso de \emph{grabar} la imagen en el disco (la memoria USB en este caso).\\

\begin{figure}
\centering
    \includegraphics[scale=0.15]{connect-camera.jpg}
    \caption{Conexi\'on de la c\'amara modular}
    \label{fig:ribbon}
\end{figure}

\subsection{Memoria USB}

Tomando en cuenta que se van a capturar fotograf\'ias una vez por hora durante 14 d\'ias se obtiene el n\'umero de im\'agenes que se deben guardar en la memoria USB: $336$. Cada fotograf\'ia tiene una resoluci\'on de \SI{1280x720}{p\'ixels}, cada p\'ixel necesita \SI{3}{B}\footnote{Uno por cada color \emph{RGB}.}, luego cada im\'agen es un archivo de \SI{2764800}{B}.\\

Dicho lo cual, necesitamos una memoria USB de \SI{1}{GB}\footnote{Exactamente \SI{928972800}{B}.}. Hace a\~nos que no se fabrican memorias USB de menos de \SI{4}{GB}; incluso se podr\'ia prescindir de la memoria USB porque el espacio que todav\'ia queda en la memoria mSD es m\'as que suficiente.\\

\subsubsection{3-2-1 backup rule}

Si no se ha cultivado un esp\'iritu aventurero es conveniente tener un \emph{backup}. Seguiremos la conocida regla \textbf{3-2-1} para pol\'iticas de backup (Figura \ref{fig:321bkp}):
\begin{center}
    \begin{description}
        \item 3 copias de las im\'agenes
        \item 2 copias en medios f\'isicos distintos
        \item 1 copia fuera del lugar f\'isico
    \end{description}
\end{center}

Resumiendo, una vez obtenida la imagen se guarda en un directorio de la tarjeta mSD, se hace una copia en la memoria USB y cada \SI{6}{h} se la sube a una carpeta alojada en un servicio cloud.

\begin{figure}
\centering
    \includegraphics[scale=0.2]{321backup.jpg}
    \caption{3-2-1 backup rule}
    \label{fig:321bkp}
\end{figure}

\subsubsection{Automontaje}

En caso de una grave falla de energ\'ia, al punto de agotar la bater\'ia del power-bank, el sistema va a reiniciar en cuanto la energ\'ia se reestablezca. Se necesita, entonces, automontar la memoria USB. Systemd mediante, esta tarea es crear un archivo \emph{unit} del tipo \emph{mount} en el directorio \texttt{/etc/systemd/system}. Es necesario nombrar esta unit de acuerdo al punto de montaje, en este caso, \texttt{mnt-bkpUSB.mount}.
\begin{lstlisting}
!\#! mkdir /mnt/bkpUSB
!\#! touch /etc/systemd/system/mnt-bkpUSB.mount
\end{lstlisting}

se busca el UUID\footnote{\emph{Universally Unique IDentifiers}, es una forma un\'ivoca de identificar un dispositivo de almacenamiento.} de la memoria USB:\\
\lstinline{# blkid}\\

para finalmente editar la unidad de montaje como a continuaci\'on:
\begin{scriptsize}
\begin{mdframed}
%\begin{shaded}
%\begin{lstlisting}
\begin{verbatim}
[Unit]
Description=USB backup
Before=local-fs.target umount.target

[Mount]
What=/dev/disk/by-uuid/...
Where=/mnt/bkpUSB
Type=ext4
Options=defaults

[Install]
WantedBy=local-fs.target
\end{verbatim}
%\end{lstlisting}
%\end{shaded}
\end{mdframed}
\end{scriptsize}

Por \'ultimo se activa la unidad, de otro modo systemd la pasar\'a por alto:\\
\lstinline{# systemctl enable mnt-bkpUSB.mount}\\

Con SysVinit hab\'ia que modificar el archivo \texttt{/etc/fstab} y agregar una l\'inea como la siguiente\footnote{La opci\'on \emph{auto} indica que se automonte.}:
\begin{scriptsize}
\begin{mdframed}
\begin{verbatim}
<file system>       <mount point>   <type>  <options>               <dump>  <pass>
UUID=73a104af-6a... /mnt/bkpUSB     ext4    defaults,noatime,auto   0       0
\end{verbatim}
\end{mdframed}
\end{scriptsize}

\section{Servicios}

Los servicios ser\'an expuestos s\'olo en su objetivo, no describiremos el c\'odigo por muy simple que sea.
\begin{itemize}
    \item tomar foto
    \item sincronizar carpeta con fotos
    \item log
\end{itemize}

\section{Timers}

Los servicios son disparados por \emph{systemd}.

\section{Servicios al inicio}

Hay servicios que deben ejecutarse al inicio, esto es, cada vez que la rpi2 bootee.

\section{Conclusi\'on}

La conclusi\'on es que systemd es m\'as complicado pero no nos queda otra.

\end{document}